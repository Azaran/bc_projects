\documentclass[11pt,a4paper,twocolumn]{article}
\usepackage[left=1.5cm,text={18cm,25cm},top=2.5cm]{geometry}
%\usepackage[IL2]
%\usepackage{czech}
\usepackage[czech]{babel}
\usepackage[utf8]{inputenc}
\usepackage{times}
\usepackage{amsthm}
\usepackage{amsmath}
\usepackage{amssymb}

\theoremstyle{definition}
\newtheorem{defn}{Definice}[section]
\theoremstyle{lemma}
\newtheorem{algo}[defn]{Algoritmus}
\newtheorem{sntc}{Věta}


\begin{document}
\begin{titlepage}
  \begin{center}

    \textsc{\Huge Fakulta informačních technologií\\Vysoké učení technické v~Brně}\\
    \vspace{\stretch{0.382}}\LARGE Typografie a publikování - 2. projekt\\Sazba dokumentů
s~matematickými výrazy\vspace{\stretch{0.618}}\\
  \end{center}
  {\Large 2015 \hfill Vojtěch Večeřa}
\end{titlepage}

\section*{Úvod}
\label{uvod}
\noindent V~této úloze si vyzkoušíme sazbu titulní strany, matematických vzorců, prostředí a
dalších textových struktur obvyklých pro technicky zaměřené texty (například rovnice \ref{rov1}
nebo definice \ref{defn1} na straně \pageref{defn1}).

\indent Na titulní straně je využito sázení nadpisu podle optického středu s~využitím zlatého řezu. Tento postup byl probírán na přednášce.

\section{Matematický text}
\label{mattext}
\noindent Nejprve se podíváme na sázení matematických symbolů a výrazů v~plynulém textu. Pro
množinu $V$ označuje card($V$) kardinalitu $V$. Pro množinu $V$ reprezentuje $V^*$ volný monoid
generovaný množinou $V$ s~operací konkatenace.
Prvek identity ve volném monoidu $V^*$ značíme symbolem $\varepsilon$.
Nechť $V^+ =V^*-\{\varepsilon\}$. Algebraicky je tedy $V^+$ volná pologrupa generovaná množinou
$V$ s~operací konkatenace. Konečnou neprázdnou množinu $V$ nazvěme abeceda.
Pro $w \in V^*$ označuje $|w|$ délku řetězce $w$ Pro $W \subseteq V$ označuje $\mathrm{occur}(w,
W)$ počet výskytů symbolů z~$W$ v~řetězci $w$ a $\mathrm{sym}(w, i)$ určuje $i$-tý symbol řetězce
$w$; například $\mathrm{sym}(abcd, 3) = c$.

\indent Nyní zkusíme sazbu definic a vět s~využitím balíku \texttt{amsthm}.
\begin{defn} 
  \label{defn1}
  \textit{Bezkontextová gramatika} je čtveřice $G = (V,T,P,S)$, kde $V$ je totální abeceda,
  $T\!\subseteq\!V$ je abeceda terminálů, $S\!\in\!(V-T)$ je startující symbol a $P$ je konečná
  množina \textit{pravidel} tvaru $q\!: A\!\rightarrow\alpha$, kde 
  $A\,\in\,(V-T)$,$\,\alpha\in\!V^*$ a $q$ je návěští tohoto pravidla. Nechť $N=V-T$ značí 
  abecedu neterminálů. Pokud $q\!:\,A\rightarrow\alpha\in\,P,\,\gamma,\,\beta\in\!V^*$, $G$ 
  provádí derivační krok z~$\gamma A\beta$ do $\gamma\alpha\beta$ podle pravidla 
  $q\!: A \rightarrow \alpha$, symbolicky píšeme 
  $\gamma A\beta \Rightarrow \gamma\alpha\beta [q\!: A \rightarrow \alpha]$ nebo zjednodušeně
  $\gamma A\beta \Rightarrow \gamma\alpha\beta$. Standardním způsobem definujeme $\Rightarrow^m$,
  kde $m \geq 0$. Dále definujeme tranzitivní uzávěr $\Rightarrow^+$ a tranzitivně-reflexivní
  uzávěr $\Rightarrow^*$.
\end{defn}

Algoritmus můžeme uvádět podobně jako definice textově, nebo využít pseudokódu vysázeného ve vhodném prostředí (například algorithm2e).


\begin{algo} 
  \label{algo1}
  Algoritmus pro ověření bezkontextovosti gramatiky. Mějme gramatiku G = (N, T, P, S).
  \begin{enumerate}
    \item\label{krok1} Pro každé pravidlo $p\in P$ proveď test, zda $p$ na levé straně obsahuje
      právě jeden symbol z~$N$.
    \item\label{krok2} Pokud všechna pravidla splňují podmínku z~kroku \ref{krok1}, tak je
      gramatika $G$ bezkontextová.
  \end{enumerate}
\end{algo}

\begin{defn} 
  \label{defn2}
  Jazyk definovaný gramatikou $G$ definujeme jako $L(G)=\{w\in T^*|S\Rightarrow^*w\}$.
\end{defn}

\subsection{Podsekce obsahující větu}

\begin{defn} 
  \label{defn3}
  Nechť $L$ je libovolný jazyk. $L$ je {\itshape bezkontextový} jazyk, když a jen když $L=
  (G)$, kde $G$ je libovolná bezkontextová gramatika.
\end{defn}

\begin{defn} 
  \label{defn4}
  Množinu $\mathcal{L}_{CF}=\{L|L$ je bezkontextový jazyk$\}$ nazýváme {\itshape třídou bezkontextových jazyků.}
\end{defn}

\begin{sntc}
  \label{sntc1}
  Nechť $L_{abc}=\{a^nb^nc^n|n\geq 0\}$. Platí, že $L_{abs}\notin \mathcal{L}_{CF}$.
\end{sntc}

\begin{proof} 
  \label{proof}
  Důkaz se provede pomocí Pumping lemma pro bezkontextové jazyky, kdy ukážeme, že není možné, aby
  platilo, což bude implikovat pravdivost věty \ref{sntc1}.
\end{proof}

\section{Rovnice a odkazy}
\label{rovniceaodkazy}

\noindent Složitější matematické formulace sázíme mimo plynulý text. Lze umístit několik výrazů
na jeden řádek, ale pak je třeba tyto vhodně oddělit, například příkazem \verb|\quad|.

$$\sqrt[x^2]{y^3_0} \quad \mathbb{N} =\{0, 1, 2,\cdots\}\quad x^{y^y} \not =x^{yy} \quad z_{i_j}
\not\equiv z_{ij}$$

\indent V~rovnici (\ref{rov1}) jsou využity tři typy závorek s~různou explicitně definovanou velikostí.

\begin{equation}\label{rov1}
 \bigg\{\Big[\big(a+b\big)*c\Big]^d+1\bigg\}=x 
\end{equation}
$$\lim_{x\to \infty} \frac{\sin{x}^2+\cos{x}^2}{4} = y$$

\indent V~této větě vidíme, jak vypadá implicitní vysázení limity $\lim_{n\to \infty}f(n)$
v~normálním odstavci textu. Podobně je to i s~dalšími symboly jako $\sum_1^n$ či
$\bigcup_{A\in \mathcal{B}}$. V~případě vzorce $\lim\limits _{x\to 0} \frac{\sin{x}}{x}\!=\!1$ jsme si vynutili méně úspornou
sazbu příkazem \verb|\limits|.

\begin{eqnarray}
  \int\limits_{a}^{b} f(x)dx & = & -\int_b^a f(x)dx \\
  \Big(\sqrt{5}{x^4}\Big)' = \Big(x^\frac{4}{5}\Big)' & = & \frac{4}{5}x^{-\frac{1}{5}} =
  \frac{4}{5\sqrt[5]{x}} \\
  \overline{\overline{A \lor B}} & = & \overline{\overline{A}\land\overline{B}}
\end{eqnarray}

\section{Matice}
\label{matice}

\noindent Pro sázení matic se velmi často používá prostředí \texttt{array} a závorky (\verb|\left|, \verb|\right|). 

\[
  \left(
  \begin{array}{cc}
  a+b & b-a \\
  \widehat{\xi+\omega} & \hat{\pi} \\
  \vec{a} & \overleftarrow{A}\!\overrightarrow{C} \\
  0 & \beta
\end{array}
\right)
\]
\[ A~= 
  \left\|
  \begin{array}{cccc}
    a_{11} & a_{12} &\cdots &a_{1n}\\
    a_{21} & a_{22} &\cdots &a_{2n}\\
    \vdots & \vdots &\ddots &\vdots\\
    a_{m1} & a_{m2} &\cdots &a_{mn}\\
  \end{array}
  \right\|
\]
\[ \left|
  \begin{array}{cc}
    t & u\\
    v~& w
  \end{array}
  \right|
  =tw-uv
\]
\indent Prostředí \texttt{array} lze úspěšně využít i jinde.

\[
  \left(
  \begin{array}{c}
    n\\
    k~\end{array}
  \right)
  = \left\{
    \begin{array}{l l}
      \frac{n!}{k!(n-k)!} & \text{pro } 0 \leq k~\leq n \\
      0 & \text{pro } k~< 0 \text{ nebo } k~> n
    \end{array}
\right. 
  \]

\section{Závěrem}
\label{zaverem}

\noindent V~případě, že budete potřebovat vyjádřit matematickou konstrukci nebo symbol a nebude
se Vám dařit jej nalézt v~samotném \LaTeX u, doporučuji prostudovat možnosti balíku maker 
\AmS-\LaTeX.
Analogická poučka platí obecně pro jakoukoli konstrukci v~\TeX u.

\end{document}
