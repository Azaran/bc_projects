\documentclass[11pt,a4paper,twocolumn]{article}
\usepackage[left=2cm,text={17cm,24cm},top=2.5cm]{geometry}

\usepackage[czech]{babel}

\usepackage[utf8]{inputenc}
\usepackage{times}

\providecommand{\myuv}[1]{\quotedblbase #1\textquotedblleft}
\renewcommand*\labelenumi{\theenumi}


\begin{document}
\title{Typografie a publikování \\ 1. projekt}
\author{Vojtěch Večeřa \\ xvecer18@stud.fit.vutbr.cz}
\date{}
\maketitle
\section{Hladká sazba}

\noindent Hladká sazba je sazba z~jednoho stupně, druhu a řezu písma sázená na stanovenou šířku odstavce. Skládá se z~odstavců, které obvykle začínají zarážkou, ale mohou být sázeny i~bez zarážky -- rozhodující je celková grafická úprava. Odstavce jsou ukončeny východovou řádkou. Věty nesmějí začínat číslicí. 

\indent Barevné zvýraznění, podtrhávání slov či různé velikosti písma vybraných slov se zde také nepoužívá. Hladká sazba je určena především pro delší texty, jako je například beletrie. Porušení konzistence sazby působí v textu rušivě a unavuje čtenářův zrak. 

\section{Smíšená sazba}

\noindent Smíšená sazba má o něco volnější pravidla, jak hladká sazba. Nejčastěji se klasická hladká sazba doplňuje o~další řezy písma pro zvýraznění důležitých pojmů. Existuje \myuv{pravidlo}:

\bigskip
\begingroup
\leftskip4ex
\rightskip\leftskip
{\scshape\indent Čím více druhů, řezů, velikostí, barev písma a~jiných efektů po\-uži\-je\-me, tím profesionálněji bude dokument vypadat. Čtenář tím bude vždy nadšen!
}\par
\endgroup
\bigskip

\indent Tímto pravidlem se \underline{nikdy} nesmíte řídit. Příliš časté zvýrazňování textových elementů a~změny {\huge V}{\LARGE E}{\Large L}{\large I}{\normalsize  K}{\small O}{\footnotesize S}{\scriptsize T}{\tiny I} \normalsize písma \Large jsou \LARGE známkou \textbf{\Huge amatérismu} \normalsize autora a~působí \textbf{\itshape velmi} \textit{rušivě}. Dobře navržený dokument nemá obsahovat více než 4 řezy či druhy písma.\texttt{ Dobře navržený dokument je decentní, ne chaotický}.

\indent Důležitým znakem správně vysázeného dokumentu je konzistentní používání různých druhů zvýraznění. To například může znamenat, že \textbf{tučný řez} písma bude vyhrazen pouze pro klíčová slova, \textsl{skloněný řez} pouze pro doposud neznámé pojmy a~nebude se to míchat. Skloněný řez nepůsobí tak rušivě a~používá se častěji. V~\LaTeX u jej sázíme raději příkazem \verb|\emph{text}| než \verb|\textit{text}|.

\indent Smíšená sazba se nejčastěji používá pro sazbu vědeckých článků a~technických zpráv. U~delších dokumentů vědeckého či technického charakteru je zvykem upozornit čtenáře na význam různých typů zvýraznění v~úvodní kapitole. 

\section{České odlišnosti}

\noindent Česká sazba se oproti okolnímu světu v~některých aspektech mírně liší. Jednou z~odlišností je sazba uvo\-zo\-vek. Uvozovky se v češtině používají převážně pro zob\-ra\-ze\-ní přímé řeči. V~menší míře se používají také pro zvýraznění přezdívek a ironie. V češtině se~použí\-vá tento \myuv{typ uvozovek} namísto anglických "uvozovek". 

\indent Ve smíšené sazbě se řez uvozovek řídí řezem prv\-ní\-ho uvozovaného slova. Pokud je uvozována celá věta, sází se koncová tečka před uvozovku, pokud se uvozuje slovo nebo část věty, sází se tečka za uvozovku. 

\indent Druhou odlišností je pravidlo pro sázení konců řádků. V~české sazbě by řádek neměl končit osamocenou jednopísmennou předložkou nebo spojkou (spojkou \myuv{a} končit může při sazbě do 25 liter). Abychom \LaTeX u zabránili v~sázení osamocených předložek, vkládáme mezi předložku a~slovo nezlomitelnou mezeru znakem \,\~\, (vlnka, tilda). Pro automatické do\-plnění vlnek slouží volně šiřitelný program \textit{vlna} od pana Olšáka\footnote{Viz ftp://math.feld.cvut.cz/pub/olsak/vlna/.}.
\end{document}

