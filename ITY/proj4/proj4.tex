\documentclass[11pt,a4paper,onecolumn]{article}
\usepackage[left=2cm,text={17cm,24cm},top=3cm]{geometry}

\usepackage[czech]{babel}
\usepackage[utf8]{inputenc}
\usepackage{times}
\pagestyle{empty}
\usepackage{natbib}        % pouzita literatura
\usepackage{url}             % URL
\DeclareUrlCommand\url{\def\UrlLeft{<}\def\UrlRight{>} \urlstyle{tt}}
\providecommand{\myuv}[1]{\quotedblbase #1\textquotedblleft}

\begin{document}
\begin{titlepage}
  \begin{center}

    \textsc{\Huge Vysoké učení technické v~Brně\\\huge Fakulta informačních technologií}\\
    \vspace{\stretch{0.382}}\LARGE Typografie a publikování -- 4. projekt\\\Huge
    Bibliografické citace\vspace{\stretch{0.618}}\\
  \end{center}
  {\Large 8. dubna 2015 \hfill Vojtěch Večeřa}
\end{titlepage}

\newpage
\pagestyle{plain}
\section*{Typografie v současnosti}

\subsection*{Co je to typografie?}
\begingroup
\leftskip4ex
\rightskip\leftskip
{\myuv{Typografie je disciplína zabývající se písmem, především jeho správným výběrem, použitím a sazbou. Cílem typografie je zajistit čtenáři snazší čtení, efektivnější vnímání čteného textu a případně i vyloučit možné chyby a nejednoznačnosti plynoucí z více možných zápisů téže věty.}\citep{Strafelda:co_je_typografie}
}\par
\endgroup 


\subsection*{Moderní formy typografie}
Počátky moderní typografie lze datovat do počátků 20. století za největšího představitele této doby je považován Karel Teige, který již od velmi útlého věku uplaťňoval principy nové typografie. \citep{Pecina:o_konstruktivismu_v_typografii}

\subsubsection*{Moderní písmo}

V současnosti existuje zhruba 400 druhů písem počínaje latinkou a končeje Braillovým písmem. Již zmíněná latinka je pravděpodobně nejrozšířenějším typem písma současnosti. Klíčem k její popularitě je její jednoduchá rozšiřit o interpunkci díky čemž vyhovuje velkému množství jazyků. \citep{Jiricek:zivy_font}

\subsubsection*{Typografie a počítače}
S vytvářením a upravovaním dokumentů se setkáváme až v době kdz se rošířily počítače a začaly obsahovat nějaké elementární programy. V daném období sloužily aplikace pouze jako simulace strojopisu a jako příprava dokumentů pro jehličkové tiskárny. O velký průlom se postaralo vytváření předloh sborníků a skript pro tisk. \citep{Rybicka:typografie_a_tex}

\subsubsection*{TeX}

Změnu v přípravě dokumentů pro tisk přinesl \TeX , který slouží pouze jako sazeč písma a neobsahuje typografická pravidla, která přínáší až jeho nástavby. Tyto nástavby ovšem stále používají \TeX jako sázeč tudíž jejich příkazy jsou transformovány na příkazy v \TeX u . \citep{Lamport:latex_a_document_preparation_system}

Nejvýznamnějším rozšířením je \LaTeX , který za cíl usnadnit práci s \TeX em lajkům a lidem, kteří se sazbou teprve začínají. V \LaTeX u uživatel neříká \textit{jak} chce věci sázet, ale \textit{co} chce sázet. Na otázku \textit{jak} odpovídá \LaTeX . \citep{Rybicka:latex_pro_zacatecniky}
\paragraph*{Další rozšíření}
Balíček \textbf{sudokubundle} je zajímavé rozšíření pro \TeX \,zprostředkovávající sazbu, generování a řešení (\textit{zjednodušování}) sudoku na určitou složitost. \citep{Wilson:sudokubundle}

\subsubsection*{Web}
Jedná se o další velké médium, kde je třeba řešit alespoň základní typografické otázky. \citep{Stastny:navrh_internetovych_stranek}
Jaké nástroje tedy můžeme použít? Základním nástrojem pro práci s webem je jazyk HTML ten ovšem není moc vhodný pokud požadujeme vytříbený typografický styl.Eliminování nedostatků HTML je ovšem jednoduché přidáme--li technologii CSS většina těchto nedostatků je eliminována. \citep{Pecina:typografie_na_webu}

Co nám tedy CSS z pohledu typografie přináší? Především možnost nastavení písma. Můžeme zde přesně definovat velikost, zarovnání, tloušťku, rodinu, variantu, styl písma atd. možností co pro nastavení parametrů písma je v CSS opravdu mnoho. \citep{Polakovic:CSS}

\newpage
\bibliographystyle{csplainnat}
\renewcommand{\refname}{Literatura}
\bibliography{literatura}

\end{document}

