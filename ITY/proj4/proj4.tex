\documentclass[11pt,a4paper,onecolumn]{article}
\usepackage[left=2cm,text={17cm,24cm},top=3cm]{geometry}

\usepackage[czech]{babel}
\usepackage[utf8]{inputenc}
\usepackage{times}
\pagestyle{empty}
\usepackage{natbib}        % pouzita literatura
\usepackage{url}             % URL

\begin{document}
\begin{titlepage}
  \begin{center}

    \textsc{\Huge Vysoké učení technické v~Brně\\\huge Fakulta informačních technologií}\\
    \vspace{\stretch{0.382}}\LARGE Typografie a publikování -- 4. projekt\\\Huge
    Bibliografické citace\vspace{\stretch{0.618}}\\
  \end{center}
  {\Large 8. dubna 2015 \hfill Vojtěch Večeřa}
\end{titlepage}

\section*{Historie typografie a tisku}
\subsection*{Tisk}
\noindent První výskyty typografie se datují do Mezopotámie, Babylonu a Řecka do doby 2000 let př.n.l. Samozřejmě se jednalo pouze o primitivní formy jako například pečetě, mince apod. 

\indent Svůj hlavní rozvoj zažívá typografie v období středověku. Z této doby bylo nalezeno několik předmětů poukazujících na to, že se již objevovali jisté typgrafické techniky. Do tétot doby se také datuje vznik pvního \uv{pohyblivého} psacího stroje sestrojeného v Číně v 11. století odkud se tato myšlenka rozšířila do Korei. Jako první byly použity keramické součásti, které ale později byly nahrazeny dřevěnými a bronzovými. Toto vyústilo až v kovový psací stroj velmi podobný tomu, kterým se proslavil Johannes Gutenberg.

\indent Gutenbergův psací stroj je ovšem pvní, který používá mechanický lis pro tisk. Jeho součásti tvořené ze slitinou kde bylo hlavní složkou olovo se osvědčily natolik, že se používají dodnes. S tím také vynalez techniku pro odlívání levných razníků písmen a jejich následné kombinování pro vytvoření více kopií. Tímto se mu v 15. století podařilo odstartovat tzv.Tiskařskou revoluci.

\indent V 20. století proběhla další revoluce typografie a to díky postupnému rozšíření osobních počítačů, které dovolili tvůrcům vytvářet a používat experimentální druhy písem a stejně tak, jako písma již zažitá. Což umožnilo novým návrhářům se prosadit.

\subsection*{Typografie}
\noindent Typografie se vyvýjela postupně s vývojem tisku a od svého počátku se značně změnila ovšem je poněkud konzervativní takže ponechává ty nejlépe čitelná písma i přesto, že se jedná písmo používané po dlouhou dobu.

\indent V raných fázích tisku v Evropě byla snaha o připodobnění tisknutého textu k tomu ručně psanému. Nejednalo se ale o úplně ideální ideu. Problém tohoto typu tisku byl že tisk nebyl lehce čitelný. Toto se ovšem začalo měnit s příchodem kovových psacích strojů ve středověku, kde začalo přecházet k použítí Římského písma. Kořeny Římského písma ovšem směřují do Řecka.

\noindent V současnoti rozlišujeme dva druhy Římského písma \uv{staromódní} a moderní. Staromódní je charakteristické tím, že je má stejně tučné čáry, kdežto moderní využívá kontrastu tenkých a širokých čar.
\newpage





   0/2 monografie (alespoň jedna musí být cizojazyčná),
   
   0/3 elektronické dokumenty,
   
   0/1 seriálovou publikaci (časopis či sborník z konference),
   
   0/2 články v seriálové publikaci,
   
   0/2 kvalifikační práce (bakalářské, diplomové nebo disertační).




\end{document}

