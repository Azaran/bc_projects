\documentclass[12pt,a4paper,onecolumn]{article}
\usepackage[left=2cm,text={17cm,24cm},top=2.5cm]{geometry}
\usepackage[czech]{babel}
\usepackage[utf8]{inputenc}
\usepackage{times}
\usepackage{setspace}
\onehalfspacing
\begin{document}
\begin{titlepage}
    \begin{center}
        \textsc{\Huge Vysoké učení technické v~Brně\\\huge Fakulta informačních technologií}\\
        \vspace{\stretch{0.382}}\LARGE Síťové aplikace a správa sítí\\
        \Huge Tunelování protokolem 6in4 se záznamem toků\\
        \large Zadavatel: Ing. Martin Kmeť\vspace{\stretch{0.618}}\\
    \end{center}
    {\Large\today\hfill Vojtěch Večeřa }
\end{titlepage}
\tableofcontents
\newpage
\section{Úvod}
Jedním z největších problémů současného internetu je nedostatek IPv4 adres. Tento problém již má
své řešení a tím je IPv6 adresace. S příchodem IPv6 se ovšem objevuje úplně nový problém, který
akutně potřebuje řešení a to je společná existence IPv6 a IPv4 na síti a zajištění jejich
vzájemného propojení. Řešení typu tak zrušíme IPv4 a bude jen IPv6 není bohužel možné a to hlavně
kvůli hardwarovým zařízením, které si s IPv6 neumí poradit. Taktéž překonfigurování všech
již vytvořených internetových sítí na IPv6 by nebylo triviální a obecně realizovatelné. Je tedy
potřeba zajistit vzájemnou kompatibilitu tam, kde je to možné. Jedním z řešeních je využití
tunelování IPv6 komunikace skrze IPv4.\\
\indent Aplikace musí umět pracovat s parametry {\it lan, wan, remote, log} a implementovat
tunelování protokolem 6in4 na základě RFC 4213 pouze s využitím BSD socketů s použítím pouze
základních a systémových knihoven jazyků C nebo C++. 

\section{Návrh aplikace}
Je třeba rozdělit aplikaci na dvě části, které spolu nikterak nekomunikují, k zajištění
zabalení IPv6 paket do paketu ve tvaru definovaném 6in4 protokolem na jedné straně tunelu a
zajištění rozbalení paketu a zformování původního IPv6 paketu na straně druhé.\\
\indent Jako vstup první části aplikace je braná odchycená komunikace na rozhraní definovaném
parametrem{\it lan}, které je připojeno do IPv6 sítě. Takto odchycenou komunikaci je potřeba
zabalit do IPv4 paketu, tedy vytvořit IPv4 hlavičku ve standardní formě definované v sekci 3.5.
RFC 4213, nastavit v ní adresu cíle na adresu uvedenou v {\it remote}, vzít přijatý paket bez
ethernetové hlavičky a připojit ho za takto vytvořenou hlavičku a předat paket
rozhraní{\it wan}.\\
\indent Vstupem druhé části je odchycená komunikace na rozhraní {\it wan}, které je rozhraním
specifikovaným jako koncový bod tunelu. Od takto odchycených paketů odstraníme ethernetovou
hlavičku a IPv4 hlavičku a tento IPv6 paket předáme na rozhraní {\it lan}.\\
\indent Výstupy aplikace je výpis na stdin a do logovacího souboru definovaného parametrem
{\it log}. Formát obou výpisů je definován zadáním a je identický.

\section{Implementační detaily}
Pro implementaci aplikace jsem si vybral jazyk C++. K rozdělení aplikace na více paralelně
fungujících částí lze využít buďto rozdělení na procesy nebo na výpočetní vlákna. Já jsem si v
tomto případě vybral rozdělení na výpočetní vlákna, kvůli jednodušímu předání společných dat oboum
vláknům a také kvůli snažšímu zpracování ukončovacího signálu SIGINT.\\
\indent Jak již jsem zmínil aplikace má pro komunikaci využívat pouze BSD sockety. Každá část
aplikace má dva sockety. Jeden je pro naslouchání na rozhraní a druhý pro odesílání zpracovaných
přijatých dat. Odposlouchávací je nastaven jako {\it SOCK\_RAW} \textbar {\it 
SOCK\_NONBLOCK} a je svázán s rozhraní, na kterém má poslouchat pomocí funkce {\it bind()}. Pro
odesílání jsem použil {\it SOCK\_RAW} v režimech {\it AF\_INET} a {\it AF\_INET6} podle toho
jestli posíláme do IPv4 nebo IPv6 sítě.\\
\indent Pro samotné čtení dat ze socketu jsem použil funkci {\it recvfrom()}, která při
nalezení čekajícího paketu ve frontě socketu načte tento paket do bufferu. Pokud je fronta prázdná
tak se daná výpočetní vlákno na 50ms uspí. Toto je pouze kvůli snížení počtu dotazů na socket,
které by mohly dopadnout stejně, tedy nebyla by načtena žádná data do bufferu.\\
\indent Odesílání dat je realizováno funkcí {\it sendto()}, které předávám buffer s nastaveným
packetem (od síťové vrstvy výše). Generování ethernetové hlavičky nechávám na kernelu.

\section{Informace o aplikaci}
Aplikace se jmenuje {\it sixtunnel}. Pro své úspěšné spuštění potřebuje chod čtyři parametry a
to následující:
\begin{itemize}
    \item {\it \--\--lan} -- definuje rozhraní, kterým je stanice připojena do IPv6 sítě
    \item {\it \--\--wan} -- definuje rozhraní, kterým je stanice připojena do IPV4 sítě
    \item {\it \--\--remote} -- definuje IP adresu druhého konce tunelu, na kterou se mají
        zabalená data posílat
    \item {\it \--\--log} -- definuje cestu k logovacímu souboru
\end{itemize}
\indent Bezpečnostní politika systémů neumožňuje jen tak někomu získávat přímý přístup k libovolné
komunikaci na rozhraních, které systém obsluhuje a je tedy zapotřebí buďto nastavit uživateli tato
speciální práva nebo, jako i v případě testovací stanice, spouštět aplikace s právy administrátora
{\it sudo}.

\subsection{Metrika programu}
\begin{itemize}
    \item Počet funkcí: 14
    \item Počet řádků: 581 + 81 = 662
    \item Staticka data: 30473 Bajtů 
    \item Velikost binárky: 307987 Bajtů = 308,0 kB
\end{itemize}

\section{Použití}
Spuštění aplikace by mělo vypadat podobně jako níže uvedený příklad spouštění. Konkrétní hodnoty
parametrů ovšem záleží na topologii sítě, do které jsou stanice připojeny a na dostupných
rozhraních na stanici, kde má aplikace běžet.
\indent Pro dosáhnutí požadované funkcionality musíme aplikaci spusti na dvouch stanicích,
které budou sloužit jako koncové body tunelu a které se jsou připojeny do obou dvou typů
sítí. Příklad spouštění (hodnoty jsou pouze demonstrační):\\ \\
\textbf{Stanice1} \\
{\it sudo ./sixtunnel \--\--lan eth0 \--\--wan eth1 \--\--remote 192.168.57.101 \--\--log gw1\_sixtunnel.log}\\
\textbf{Stanice2} \\
{\it sudo ./sixtunnel \--\--lan eth0 \--\--wan eth1 \--\--remote 192.168.57.100 \--\--log gw2\_sixtunnel.log}


\section{Literatura}
\end{document}
